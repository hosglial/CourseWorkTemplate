\newpage
\anonsection{ВВЕДЕНИЕ}
В процессе эксплуатации зданий и сооружений существует потребность в мониторинге и управлении состоянием различными инженерных систем здания, таких как:

\begin{itemize}
	\itemsep0em 
	\item Водоснабжение и водоотведение
	\item Электроснабжение
	\item Газоснабжение
	\item Отопление
	\item Вентиляция
	\item Пожаротушение
	\item Кондиционирование
	\item Контроль и управления доступом и др.
\end{itemize}

Для гарантии качественного обслуживания данных систем используются системы автоматизированного мониторинга. Данные комплексы позволяют в кратчайшие сроки обнаружить и устранить неполадки в вышеуказанных системах, а также модифицировать их таким образом, чтобы они были способны автоматически реагировать на определенные внештатные ситуации, уведомлять об этом оператора и предпринимать действия по их ликвидации.

В текущей ситуации разработка подобных систем осложнена в связи с прекращением поставок оборудования и лицензий на программное обеспечение из большинства стран.

Альтернативой коммерческому ПО является Open Source программное обеспечение, распространяющееся на бесплатной основе, и открытое для модификации.
В качестве альтернативного оборудования может быть использована платформа Arduino.
Arduino - программно-аппаратная платформа для построения электронных схем, создания моделей, а также автоматизации процессов и робототехники.

Основное преимущество использования данной платформы - широкие возможности для расширения с помощью готовых модулей, либо компонентов, требующих самостоятельной сборки.

Главный недостаток - отсутствие готовых решений, из чего следует необходимость самостоятельной разработки ПО и сборки модулей, готовых для развертывания на площадке клиента.

В связи с этим требуется разработка решения, которое позволит комбинировать различные доступные модули с целью создания систем управления различной сложности.

Основные задачи курсовой работы:
\begin{itemize}
	\itemsep0em 
	\item Произвести анализ текущей ситуации и решений, представленных на рынке
	\item Произвести анализ возможных внештатных ситуаций, требующих немедленного реагирования
	\item Разработать макет технического и программного решения
	\item Привести варианты практического применения разработанных решений 
\end{itemize}


\newpage\section{Общие сведения о разрабатываемой системе}
\subsection{Назначение и цели создания системы}
Основным назначением системы является модификация существующих системы с целью осуществления автоматизированного мониторинга и управления.

Цели создания системы:
\begin{enumerate}
	\itemsep0em 
	\item Повышение надежности и отказоустойчивости существующих систем зданий и сооружений
	\item Снижение ущерба от возможных аварий
	\item Повышение скорости реагирования на внештатные ситуации
	\item Осуществление постоянного мониторинга состояния основных систем ЗиС
	\item Замена существующих импортных систем, требующих поддержки и сопровождения на собственную разработку
\end{enumerate}

\section{Требования к системе}
\subsection{Требования к системе в целом}
\begin{enumerate}
	\itemsep0em 
	\item Постоянная работа 
	\item 
	\item 
	\item 
\end{enumerate}
\subsection{Требования к численности и квалификации персонала системы}
\subsection{Требования к надежности}
\subsection{Требования к безопасности}


\newpage\section{Глава 3}






\newpage\anonsection{ЗАКЛЮЧЕНИЕ}