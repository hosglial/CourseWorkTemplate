\newpage
\anonsection{ВВЕДЕНИЕ}
В процессе эксплуатации зданий и сооружений существует потребность в мониторинге и управлении состоянием различныхи инженерных систем здания, таких как:

\begin{itemize}
	\itemsep0em 
	\item Водоснабжение и водоотведениe
	\item Электроснабжениe
	\item Газоснабжениe
	\item Отоплениe
	\item Вентиляция
	\item Пожаротушение
	\item Кондиционированиe
	\item Контроль и управления доступом и др.
\end{itemize}

Для гарантии качественного обслуживания данных систем используются системы автоматизированного мониторинга. Данные комплексы позволяют в кратчайшие сроки обнаружить и устранить неполадки в вышеуказанных системах, а также модифицировать их таким образом, чтобы они были способны автоматически реагировать на определенные внештатные ситуации, уведомлять об этом оператора и предпринимать действия по их ликвидации.

Основные задачи:
\begin{itemize}
	\itemsep0em 
	\item Произвести анализ текущей ситуации и решений, представленных на рынке
	\item Произвести анализ возможных внештатных ситуаций, требующих немедленного реагирования
	\item Разработать макет технического и програмного решения
	\item Привести варианты практического прмиенения 
\end{itemize}


\newpage\section{Анализ текущей ситуации}

На данный момент наибольшей проблемой является отстутствие иностранного оборудования и программного обеспечения. Большинство решений, представленных на рынке используют технологии Intel/Amd, а также ПО, предоставляющееся исключительно на коммерческой основе.

По этой причине следует максимально ограничить использование оборудования и ПО стран, остановивших поставки и использовать имеющиеся альтернативы.

Такими альтернативами могут стать Open Source програмное обеспечение и аппаратная платформа Arduino.

\newpage\section{Глава 2}

\newpage\section{Глава 3}

\newpage\anonsection{ЗАКЛЮЧЕНИЕ}