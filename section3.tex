\subsection{Архитектура системы}

\subsection{Базовый модуль}
Для мониторинга состояний систем зданий и сооружений предполагается использование различных датчиков, присоединенных к стандартному модулю связи и управления.
Модуль имеет широкие возможности гибкой конфигурации для различных сценариев использования.

В основе модуля лежит плата Wemos D1 Mini, имеющая следующие характеристики:
\begin{itemize}
    \item Микроконтроллер: ESP8266.
    \item Разрядность: 32 бит.
    \item Напряжение питания платы: 3,3 / 5,0 В.
    \item Беспроводной интерфейс: Wi-Fi 802.11 b/g/n 2,4 ГГц (STA/AP/STA+AP, WEP/TKIP/AES, WPA/WPA2).
    \item Поддерживаемые шины: SPI, I2C, I2S, 1-wire, UART, UART1, IR Remote Control.
    \item Цифровые выводы I/O: 11 (RX, TX, D0...D8) все выводы кроме D0 поддерживают INT (внешнее прерывание), ШИМ, I2C, 1-wire.
    \item Аналоговые входы: 1 (A0) 10-битный АЦП.
    \item Логические уровни выводов I/O: 3,3 В
    \item Максимальный ток на выводе I/O: 12 мА (для каждого вывода).
    \item Максимальное напряжение на входе A0: 3,2 В (между выводом A0 и GND)
    \item Flash-память: 4 МБ.
    \item RAM-память данных: 80 КБ.
    \item RAM-память инструкций: 32 КБ.
    \item Тактовая частота микроконтроллера: 80 МГц.
    \item Чип USB-UART преобразователя: CH340G .
    \item Рабочая температура: -40 +85 °C .
    \item Габариты: 34,2x25,6 мм.
    \item Вес: 10 г.
\end{itemize}
Главное преимущество данной платы - наличие wi-fi модуля, способного работать на частоте 2.4 Ггц непосредственно на плате, что позволяет уменьшить размеры модуля, так как дополнительный модуль связи не требуется.
Плата имеет 11 цифровых и 1 аналоговый вход, также способна работать с датчиками/модулями, поддерживающими протокол I2C .
\image{wemos_pinout.jpg}{Распиновка платы Wemos D1 Mini}{0.7}
Модуль для функционирования требует постоянное напряжение 5V, которое может быть обеспечено встроенным в плату блоком питания, преобразующим 220V AC в 5V DC

В качестве стандартного блока питания базового модуля используется HLK-PM01 - миниатюрный блок питания, обеспечивающий преобразование 220V AC в 5V DC, и обеспечивающий питание Wemos, а также подключенных модулей.
Максимальная нагрузка на данный блок питания составляет 0.6 A. Данной мощности достаточно для подключения большого количества датчиков, так как зачастую они потребляют не больше 1 мА.
\image{power_supply.png}{Блок питания HLK-PM01}{0.7}

Для удобства сборки и дальнейшего использования проектируется печатная плата.
На ней расположены модули контроллера и блока питания, а также имеются троки входов Vcc Gnd и Logic для подключения датчиков.
К базовой версии платы возможно подключить 6 цифровых/аналоговых датчиков.
При необходимости конфигурация платы может изменяться для обеспечения возможности подключения большего числа датчиков, либо датчиков/модулей с более сложными интерфейсами, например I2C .

Печатная плата спроектирована и разведена в редакторе EasyEDA .
\image{main_pcb_scheme.png}{Схема разводки печатной платы базового модуля}{0.7}
\image{main_pcb_image.png}{Изображение печатной платы базового модуля}{0.7}

\subsection{Подключаемые датчики}
Для осуществления процесса мониторинга используются цифровые/аналоговые датчики, подключаемые к базовому модулю.
Информация с датчиков проходит первичную обработку, преобразования и отправляется на главный сервер для дальнейшего вывода в терминалы доступа, на информационные экраны, а также данная информация используется для отправки команд различным управляющим устройствам.
Конфигурация датчиков зависит от конкретной ситуации.
Далее представлены основные датчики и сенсоры, осуществляющие мониторинг, и их технические характеристики.

\textbf{Датчик температуры DS18b20}
\image{ds18b20.png}{Датчик температуры}{0.7}

Описание:

DS18B20 – цифровой термометр, способный измерять температуру в диапазоне от -55оС до +125оС с программируемой точностью 9--12 бит.
При изготовлении на производстве, каждому датчику присваивается свой уникальной 64-битный адрес, а обмен информацией с ведущим устройством (микроконтроллером или платой Arduino) осуществляется по шине 1-wire.
Есть возможность подключения множества датчиков, вплоть до 2^{64}.

Технические характеристики:

\begin{itemize}
    \itemsep0em
    \item Напряжение питания: 3V-5.5V;
    \item Протокол обмена данными: 1-Wire;
    \item Способ подключения: прямой / по одной линии с паразитным питанием;
    \item Разрешение преобразования температуры: 9 бит – 12 бит;
    \item Диапазон измерения температуры: от -55 до +125 оС;
    \item Период измерения температуры при максимальной точности 12 бит: 750 мС;
    \item Тип индексации на линии 1-Wire: уникальный 64-битный адрес;
\end{itemize}


\textbf{Датчик абсолютного давления, температуры и влажности BME280}
\image{bme280.png}{Датчик давления, температуры и влажности}{0.7}

Описание:

Датчик позволяет измерять атмосферное давление, влажность и температуру, применяется в случаях, когда требуется полноценный мониторинг микроклимата в помещении.

Технические характеристики:

\begin{itemize}
    \itemsep0em
    \item Напряжение питания: 3.3 В – 5 В
    \item Рабочий ток: 1 мA
    \item Диапазон измерения давления: 300 гПа – 1100 гПа (точность ±1.0 гПа)
    \item Диапазон измерения температуры: -40 °C до +85 °C (точность ±0.5 °C)
    \item Диапазон измерения влажности: 20 \% до 80 \% (точность ±3 \%)
    \item Интерфейс: I2C
    \item Габариты: 12 мм х 10 мм
\end{itemize}


\textbf{Датчик света LM393}
\image{lm393.png}{Датчик освещенности}{0.7}

Описание:

Данный модуль используется для определения уровня освещенности помещений, либо конкретных точек.
В сочетании с управляющим устройством позволяет поддерживать заданный уровень освещенности.

Технические характеристики:
\begin{itemize}
    \itemsep0em
    \item Напряжение питания: +3.3 В ~ +5.5 В
    \item Потребляемый ток: 15 мА
    \item Формат сигнала цифрового выхода: TTL(0/1)
    \item Уровень сигнала аналогового выхода: 0..Vcc
    \item Подключается непосредственно к микроконтроллеру
    \item Рабочая температура: от 0 ° C ~ + 70 ° C
    \item Размеры: 32 x 14 мм
    \item Вес модуля: 3 грамма
    \item Диаметр монтажного отверстия: 3 мм
\end{itemize}


\textbf{Датчик препятствий YL-63}
\image{yl63.jpg}{Датчик препятствий}{0.7}

Описание:

Цифровой инфракрасный датчик обхода препятствий YL-63 (или FC-51) применяется тогда, когда нужно определить наличие объекта, а точное расстояние до объекта знать необязательно.
Датчик состоит из инфракрасного излучателя, и фотоприемника.
ИК источник излучает инфракрасные волны, которые отражаются от препятствия и фиксируются фотоприемником.
Датчик обнаруживает препятствия в диапазоне расстояний от нуля до установленной предельной границы.
Построен на основе компаратора LM393, который выдает напряжение на выход по принципу: обнаружено препятствие –логический уровень HIGH, не обнаружено – логический уровень LOW, данное состояние показывает и находящийся на датчике красный светодиод.
Пороговое значение зависит от настройки датчика и регулируется с помощью установленного на модуле потенциометра.
Для индикации питания на датчике установлен зеленый светодиод.

Технические характеристики:
\begin{itemize}
    \itemsep0em
    \item напряжение питания: 3.3–5 В
    \item тип датчика: диффузионный
    \item компаратор: LM393
    \item расстояние обнаружения препятствий: 2 – 30 см
    \item эффективный угол обнаружения препятствий: 35°
    \item потенциометр для изменения чувствительности
    \item светодиод индикации питания
    \item светодиод индикации срабатывания
    \item размеры: 43 х 16 х 7 мм
\end{itemize}

\newpage
\textbf{Датчик уровня жидкости GSMIN HW-038}
\image{water_sensor.png}{Датчик уровня жидкости}{0.7}

Описание:

Применяется в случаях, когда необходимо отслеживать наличие жидкости или конкретный её уровень, таких как затопление подвальных помещений, различные протечки и т.д.

Технические характеристики:
\begin{itemize}
    \itemsep0
    \item Рабочее напряжение: DC3-5V
    \item Рабочий ток: менее чем 20 мА
    \item Тип сенсора: аналоговый
    \item Зона обнаружения: 40 мм x 16 мм
    \item Рабочая температура: 10--30
    \item Влажность: 10\%-90\% без конденсации
    \item Размеры: 62 мм x 20 мм x 8 мм
\end{itemize}




%\codefromfile{wifi.c}{text}


\subsection{Конфигурация сервера}

%конфигурация сервера, принимающего/отправляющего информацию с датчиков/устройств, а также требуемое ПО

\subsection{Протоколы передачи данных}

%Описание http протокола, посредством которого осуществляется связь
