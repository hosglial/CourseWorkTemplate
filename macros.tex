% Список пользовательских команд 

%%%%%%%%%%%% \image %%%%%%%%%%%%%%%%%%

% \image {Имя изображения.расширение}{Подпись к рисунку}{Скейл Изображения}

\newcommand{\image}[3]{
\begin{figure}[!htb]
	\centering
	\includegraphics[width=#3\textwidth]{#1}
	\caption{#2}
\end{figure}
}


%%%%%%%%%%%%%%%%%%%%%%%%%%%%%%%%%%%%%%



%%%%%%%%%% \codefromfile  %%%%%%%%%%%%%%%%%%

% \codefromfile {Имя файла}

\newcommand{\codefromfile}[2]{
\begin{code}
	\inputminted[breaklines=true, framesep=10pt, fontsize=\footnotesize, firstline=1,]{#2}{Listings/#1}
\end{code}
}


%%%%%%%%%%%%%%%%%%%%%%%%%%%%%%%%%%%%%%


%%%%%%%%%% \anonsection  %%%%%%%%%%%%%%%%%%
%Секция, присутсвующая в оглавлении, но без номера


\newcommand{\anonsection}[1]{\section*{#1}\addcontentsline{toc}{section}{#1}}